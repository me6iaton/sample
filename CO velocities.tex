\subsubsection{CO Velocities}
\label{CO}
CO observations trace gas with mean density around 100 \cc.   CO emission associated with the Scutum-Centaurus Arm of the Milky Way is shown in Figure \ref{fig:COarm}, which presents a plane-of-the-sky map integrated over  $-50 <v_{LSR}< -30$ \kms.  The velocity range is centered on  -40 \kms, the average velocity of the Scutum-Centaurus Arm in Nessie's longitude range (see Figures \ref{fig:topview} and \ref{fig:coloredlines}).  The white chalk line superimposed on Figure \ref{fig:COarm} is the same tracing of ``Nessie Optimistic" shown in Figure \ref{fig:FindingChart}.  The black feature labeled ``Nessie" refers to ``Nessie Classic."   

The vertical (latitude) centroid of the CO emission  attributed to the Scutum-Centaurus Arm \citep{Dame2011} shown in Figure \ref{fig:COarm} appears to follow Nessie remarkably well, even out to the full $8^\circ$ (430 pc) extent of Nessie Optimistic.  Table 1 estimates Nessie's typical \htwo\ column density at $\simgreat 10^{23}$ \cmsq\ and its typical volume density at $\simgreat 10^5$ \cc.    Thus, the plane-of-the-sky coincidence of the line-of-sight-velocity-selected ``Scutum-Centaurus" CO emission  and the mid-IR extinction suggests that the Nessie IRDC may be a kind of ``spine" or ``bone" of this section of the Scutum-Centaurus Arm.   But, the spatial resolution of the CO map is too low ($8'$), and the 20 \kms\ velocity range associated with the Arm in CO is too broad to decide based on this evidence alone whether Nessie is a well-centered ``spine" or just a long skinny feature associated with, but potentially significantly inclined to, the Scutum-Centaurus Arm.   