\section{Introduction}

Determining the structure of the Milky Way, from our vantage point within it, is a perpetual challenge for astronomers. We know the Galaxy has spiral arms, but it remains unclear exactly how many, cf. \citep{Vallee2008a}. Recent observations of maser proper motions give unprecedented accuracy in determining the three-dimensional position of the Galaxy's center and rotation speed \citep{Reid2009,Brunthaler2011}. But, to date, we still do not have a definitive picture of the Milky Way's three dimensional structure.

The analysis offered in this paper suggests that some Infrared Dark Clouds\footnote {The term ``Infrared Dark Cloud" or ``IRDC" typically refers to any cloud which is opaque in the mid-infrared.}--in particular very long, very dark, clouds--appear to delineate major features of our Galaxy as would be seen from outside of it. In particular, we study a $>3^{\circ}$-long cloud associated with the IRDC called ``Nessie" \citep{Jackson2010}, and we show that it appears to lie parallel to, and no more than just few pc from, the true Galactic Plane.

Our analysis uses diverse data sets, but it hinges on combining those data sets with a modern understanding of the meaning of Galactic coordinates. When, in 1959, the IAU established the current system of Galactic $(l,b)$ coordinates \citep{Blaauw1959}, the positions of the Sun with respect to the ``true" Galactic disk, and of the Galactic Center, were not as well determined as they are now. As a result, the Galactic Plane is typically \textbf{not} at $b=0$, as
projected onto the sky. The exact offset from $b=0$ depends on distance, as we explain in \S \ref{lookingdown}. Taking these offsets into account, one can profitably re-examine data relevant to the Milky Way's 3D structure.  The Sun's vantage point slightly ``above" the plane of the Milky Way offers useful perspective.

``IRDCs" are loosely defined as clouds with column densities high enough to be obvious as patches of significant extinction against the diffuse galactic background mid--infrared wavelengths.  \citet{Peretto2009a} set the boundaries of IRDCs at an optical depth of 0.35 at $8~\rm{}\mu{}m$ wavelength, equivalent to an $\rm{}H_2$ column density $\approx{}10^{22}~\rm{}cm^{-2}$. In the \citet{peretto2010:irdcs-mass-density} sample, clouds have average column densities of a few $10^{22}~\rm{}cm^{-2}$. Some IRDCs actively form high--mass stars (e.g., \citealt{pillai2006:g11} and \citealt{rathborne2007:irdc-msf}). \citet{kauffmann2010:irdcs} explain that while some ``starless" IRDCs are potential sites of future high--mass star formation, and the few hundred densest and the most massive, IRDCs may very well contain a large fraction of the star--forming gas in the Milky Way, it is still true that most IRDCs are not massive and dense enough to form high--mass stars. Thus, a small number of very dense and massive IRDCs may be responsible for a large fraction of the galactic star formation rate, and an extragalactic observer of the Milky Way might ``see" IRDCs not unlike Nessie hosting young massive stars as the predominant mode of star formation here.  

The traditional ISM-based probes of the Milky Way's structure have been HI and CO. Emission in these tracers gives line intensity as a function of velocity, so the position-position-velocity data resulting from HI and CO observations can give three dimensional views of the Galaxy, if a rotation curve is used to translate line-of-sight velocity into a distance. Unfortunately, though, the Galaxy is filled with HI and CO, so it is very hard to disentangle features when they overlap in velocity along the line of sight. Nonetheless, much of the basic understanding of the Milky Way's spiral structure we have now comes from HI and CO observations of the Galaxy, much of it from the compilation of CO data presented by \citet{Dame2001}.

Recently, several groups have targeted high-mass star-forming regions in  the plane of the Milky Way for high-resolution observation. In their BeSSeL Survey, Reid et al. are using hundreds of hours of VLBA time to observe hundreds of regions for maser emission, which can give both distance and kinematic information for very high-density ($n>10^8$ cm$^{-3}$) gas \citep{Reid2009,Brunthaler2011}. In the HOPS Survey, hundreds of positions associated with the dense peaks of infrared dark clouds have now been surveyed for ${\rm NH}_3$ emission \citep{Purcell2012b}, yielding high-spectral resolution velocity measurements towards gas whose density typically exceeds $10^4$ cm$^{-3}$. In follow-up spectral-line surveys to the ATLASGAL \citep{Beuther2012a} dust-based survey of the Galactic Plane, \citet{Wienen2012} have measured ${\rm NH}_3$ emission in nearly 1000 locations. The ThrUMMs Survey aims to map the entire fourth quadrant of the Milky Way in CO and higher-density tracers \citep{BarnesPeter2010}, and it should yield additional high-resolution velocity measurements.

Targets in high-resolution (e.g. BeSSeL) studies are usually identified based on continuum surveys, which show the locations of the highest column-density regions, either as extinction features (``dark clouds" in the optical, ``IRDCs" in the infrared), as dust emission features (in surveys of the thermal infrared), or as gas emission features (e.g. HII regions).

Great power lies in the careful combination of continuum and spectral-line data when one wants to understand the structure of the ISM in three-dimensions. Thus, there have already been several efforts to combine dust maps with spectral-line data, whose goal is often the assignment of more accurate distances to particular clouds or regions e.g. \citep{Foster2012}.  These improved distances allow for more reliable conversion of  measured quantities (e.g. fluxes) to physical ones (e.g. mass).

In this study, our aim is to combine morphological information from large-scale mid-infrared continuum ``dust" maps of the Galactic Plane with spectral-line data, so as to understand the nature of very long infrared dark clouds that appear parallel to the Galactic Plane. We focus in particular on the IRDC named ``Nessie" in the study presented by \citet{Jackson2010}.  In that 2010 paper, Nessie is shown to be a highly elongated filamentary cloud (see Figure \ref{fig:FindingChart})  exhibiting the after-effects of a sausage instability that led to several massive-star-forming peaks spaced at regular intervals.
We extend the work of Jackson et al. by first by literally ``extending" the cloud, to a length of at least 3 degrees (\S \ref{longer}). In \S \ref{3D}, we show that a careful accounting for the modern measures of
the Sun's height off of the Galactic mid-plane and of the true position of the Galactic Center imply that Nessie lies not
just parallel to the Galactic Plane, but \emph{in} the Galactic Plane. We consider what velocity-resolved measures of the
material associated with Nessie tells us about its three-dimensional position in the Galaxy, and we conclude, in \S \ref{spine} that Nessie likely marks the ``spine" of the Scutum-Centaurus arm of the Milky Way in which it lies. In \S \ref{future}, we consider, in the light of coming computational and observational capabilities, the likelihood of finding more ``Nessie-like" structures in the future, and of using them, in conjunction with the Sun's vantage point just above the mid-plane of the Milky Way, to map out the skeleton of our Galaxy.
