\subsection{Using Rotation Curves and Velocity Measurements to Place Nessie in 3D}

Ever since velocity-resolved observations of stars and gas have been possible, astronomers have been modeling the rotation pattern of the Milky Way.   Using a measured rotation curve for the Milky Way's gas, \citep[e.g.][ ]{McClureGriffiths2007}, one can translate observed LSR velocities to a unique distance in the Outer Galaxy, and to one of two possible (``Near" or ``Far") distances toward the Inner Galaxy.   Figure \ref{fig:topview} shows iso-$v_{LSR}$ contours toward the Inner Galaxy, around the longitude range of Nessie, superimposed on the data-driven cartoon of our current understanding of the Milky Way's structure.  Notice that velocities associated with the near-side of the Scutum-Centaurus Arm in Nessie's longitude range should be near 40 km\ s$^{-1}$.

Combining a modern estimate for the Sun's height above the plane ($z_{\rm Sun}\sim 25$ pc), with the IAU Galactic coordinate definitions, we can determine where the physical mid-Plane of the Galaxy \textit{should} appear in the $(l^{II}, b^{II})$ system at any particular distance from the Sun.  Figure \ref{fig:coloredlines} shows where the Scutum-Centaurus Arm would appear on the Sky (for a distance to SrgA* of 8.5 kpc, a rotation speed for the Milky Way of 220 km\ s$^{-1}$, and (U,V,W) motion for the Sun of 11.1, 12.4, and 7.2 km\ s$^{-1}$, respectively).   As its caption explains in detail, Figure \ref{fig:coloredlines}'s colored lines are associated with the near part of the Scutum-Centaurus Arm. Two versions of this plane-of-the-Sky view are shown, one \textit{only} accounting for the offset of the Sun, and the other also accounting for the tilt of the coordinate system caused by the Galactic Center also not lying in the IAU plane (see Figure \ref{fig:galcoords}). 

The dashed colored lines in Figure \ref{fig:coloredlines}, indicating the predicted position of the Galactic Plane on the Sky at the distance to the near side of the Scutum-Centaurus Arm, pass almost directly through Nessie, regardless of whether or not one considers the ``tilt" of the coordinate system caused by SgrA*'s offset.  Solid colored lines show 20 pc above and below the Plane at the distance to the Scutum-Centaurus Arm, so Figure \ref{fig:coloredlines} makes it is very clear that Nessie lies within just a few pc of the Plane, along its entire length. This is either an extremely fortuitous coincidence, or an indication that Nessie is tracing a significant feature that effectively marks the mean location of the Galactic Plane.  Given the waviness of the plane on 10 pc scales (see above, \citep{Malhotra1994}), the location at even less than 10 pc from the mean plane is likely fortuitous--but the location so close to the mean is not.