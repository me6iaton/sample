\subsection{Contributions}

This paper was a truly a group effort, and the author list includes only
some of the many people who have contributed to it. The entire project
was inspired by a question: ``Is Nessie parallel to the Galactic Plane?,"
asked by Andi Burkert at the 2012 Early Phases of Star Formation
(EPoS) meeting at the Max Planck Society's Ringberg Castle in Bavaria. Three EPoS attendees beyond the author list contributed significant ideas and data to this
work, most notably Steven Longmore, Eli Bressert, and Henrik Beuther. We are grateful to
Cormac Purcell for giving us advance online access to the HOPS data, and
to Mark Reid for generously sharing his expertise on Galactic structure.
The text here was largely written by Alyssa Goodman; the theoretical
ideas come primarily from Andi Burkert; and much of the geometrical
analysis was carried out by Christopher Beaumont, Bob Benjamin, and Tom
Robitaille. Tom Dame and Bob Benjamin provided expertise on Galactic
structure, and they created several of the figures shown here. Jens Kauffmann provided expertise on IRDCs, and also was first to point out the potential relevance of the Sun's non-zero height above the Galactic Plane.  Joao Alves provided expertise on the potential for using extinction maps to find more Nessie-like features, and Michelle Borkin was instrumental in early visualization work that led to our present proposals for using the Sun's ``high" vantage point to map out the Milky Way.  Jim Jackson contributed critical expertise on the Nessie IRDC, based both on the 2010 study he led and on unpublished work since.

The article you are reading now was the first to be prepared using a new online collaborative authoring system called Authorea.  The early drafts of the paper, as well as the final version, were, and are, all open to the public. We thank Authorea's founders and developers, Alberto Pepe, Nathan Jenkins, and Eli Bressert, for assistance as the work proceeded. 

A.B. acknowledges support from the Cluster of Excellence ``Origin and Structure of the Universe." A.G. and C.B. thank Microsoft Research, the National Science Foundation (AST-0908159) and NASA (ADAP NNX12AE11G) for their support. M.B. was supported by the Department of Defense through the National Defense Science & Engineering Graduate Fellowship (NDSEG) Program. R.B. acknowledges NASA grant NNX10AI70G.

\subsection {Facilities}
Data in this paper were taken with the following telescopes.  The CO Survey of the Milky Way data \cite{2001ApJ...547..792D} are from the 1.2-Meter Millimeter-Wave Telescope in Cambridge, Massachusetts, USA. NH$_3$ observations of cores \cite{2012MNRAS.426.1972P} in and near Nessie are from the Mopra 22-meter telescope near Coonabarabran, Australia. The mid-infrared images of the Galactic Plane used to define Nessie are from NASA's Spitzer Space Telescope, and they were made as part of the Galactic Legacy Infrared Mid-Plane Survey Extraordinaire (GLIMPSE\cite {2003PASP..115..953B},\cite{2009PASP..121..213C}) and MIPSGAL (where MIPS=Multiband Infrared Photometer for Spitzer (MIPS))\cite {http://adsabs.harvard.edu/cgi-bin/nph-data_query?bibcode=2009PASP..121...76C&db_key=AST&link_type=ABSTRACT} Surveys of the Galactic Plane.  