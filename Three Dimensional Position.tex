\section{The Three-Dimensional Position of Nessie within the Milky Way}
\label{3D}

\subsection{Looking ``Down" on the Galaxy}
\label{lookingdown}
Astronomers would love to travel far beyond the Milky Way, so that we could observe its spiral pattern face-on, as we do for other galaxies.  But our Sun is so entrenched in the Milky Way's plane that an ``overhead view" of the Milky Way's structure is impossible.  Or is it?  What if the Sun were just far enough above the Galactic Plane that we could use its height to give ourselves a tiny bit of perspective on the Galactic Plane that would spatially separate features--if they are very narrow--at different distances to be at different projected latitudes?  Turns out we {\it are} lucky in this way--the Sun {\it is} apparently located a bit above the Plane (see below), and we can use that vantage point to out advantage.  

To understand why most astronomers do not yet consider the possibility or value of an overhead view, we need to consider the origin of our current Galactic coordinate system, and our current understanding of the Sun's and the Galactic Center's 3D positions.  Writing in 1959 on behalf of the International Astronomical Union's (IAU's) sub-commission 33b, Blaauw et al.
wrote: 
\begin{quotation}
The equatorial plane of the new co-ordinate system must of
necessity pass through the sun. It is a fortunate circumstance that,
within the observational uncertainty, both the sun and Sagittarius A lie
in the mean plane of the Galaxy as determined from hydrogen
observations. If the sun had not been so placed, points in the mean
plane would not lie on the galactic equator.
\end{quotation}
In a further explanation of the IAU system in 1960, Blaauw et al. explain that stellar observations did, at that time, indicate  the Sun to be at $z_{\rm Sun}=22 \pm 2$ (22 pc above the plane), but the authors then discount those observations as too dangerously affected by hard-to-correct-for extinction in and near the Galactic Plane \citep{1960MNRAS.121..123B}.   Instead, the 1959 IAU system relies on the 1950's measurements of HI, which showed the Sun to be at $z_{\rm Sun}=4\pm 12$ pc off the Plane, consistent with the Sun being directly in the Plane ($z_{\rm Sun}=0$).  Interestingly, since the 1950's, the Milky Way's HI layer has been shown to have corrugations on the scale of 10's of pc \citep{Malhotra1995}, and there may be similar fluctuations in the mid-plane of the ${\rm H_2}$ \citep{Malhotra1994}, so it is still tricky to use gas measurements to determine the Sun's height off the plane.

Astronomers today are still using the $(l^{II}, b^{II})$ Galactic coordinate system defined by
\citet{Blaauw1959}, but it is \emph{not} still the case, within observational uncertainty, that the
Sun is in the mean plane of the Galaxy, and the true position of the Galactic Center is no longer at $(l^{II}=0, b^{II}=0)$.
Instead, a variety of lines of evidence \citep{Chen2001b,Maiz-Apellaniz2001a,Juric2008a} show that the Sun is approximately 25 pc above the stellar Galactic mid-plane, and VLBA proper motion observations of masers show that the Galactic Center is about 7 pc below where the $(l^{II}, b^{II})$ system would put it, at $b=-0.046^\circ$ \citep{Reid2004}.   These offsets, as predicted by Blaauw et al., imply that ``points in the mean plane [do] not lie on the galactic equator."

Figure \ref{fig:galcoords} shows a schematic (not-to-scale) diagram of the effect of the Sun's and the Galactic Center's offsets from the  mid-plane defined by the IAU in 1959 (and still in use as $(l^{II}, b^{II})$ today).  The tilt of the the true, physical, Galactic mid-plane to the presently IAU-defined plane means that, within about 12 kpc of Sun\footnote{12 kpc is the approximate distance where the physical and IAU planes cross, on a line toward the Galactic Center.  Along other directions toward the Inner Galaxy, as shown in the lower panel of Figure \ref{fig:topview}, it will be further to the crossing point, and toward the Outer Galaxy, for a ``flat" disk, the mid-plane will always appear at negative latitudes.} any feature that is truly ``in" the Galactic mid-plane will appear on the Sky at negative $b^{II}$.  Figure \ref{fig:coloredlines} shows an example of this effect, where the rainbow-colored dashed line indicates the sky position of the physical Galactic mid-plane at a Nessie-like distance of 3.1 kpc (assuming the the Sun is 25 pc off the plane, a distance to SrgA* of 8.5 kpc, a rotation speed for the Milky Way of 220 km\ s$^{-1}$, and (U,V,W) motion for the Sun of 11.1, 12.4, and 7.2 km\ s$^{-1}$, respectively).