\section{Nessie is Longer than We Thought}
\label{longer}

Nessie was discovered and named 
using Spitzer Space Telescope images that show the cloud as a very clear absorption feature at mid-infrared wavelengths
\cite{Jackson2010}.  Using observations of the dense-gas tracer HNC, Jackson
et al. (2010) further show that the section of the cloud from $l=337.85$ to $339.1$
(labelled ``Nessie Classic" in Figure \ref{fig:FindingChart}) exhibits very similar line-of-sight
velocities, ranging over $-40<v_{LSR}<-36$ km\ s$^{-1}$.
The similarities of these line-of-sight velocities is taken to mean that
the cloud is a coherent, long structure, and not a chance
plane-of-the-sky projection of disconnected features. Thus, ``Nessie
Classic" is shown to be a dense,  long ($\sim 1^\circ$), narrow ($\sim 0.01^\circ$),
filament, and Jackson et al. (2010) ultimately conclude that it is undergoing a
sausage instability leading to density peaks hosting active sites of
massive star formation.

Our purpose in looking at Nessie again here is not to further analyze the star-forming nature of this cloud. Instead, our focus is on how long the full Nessie feature might be, and what its length and its three-dimensional position might imply about its role in the Galaxy. Casual inspection of Spitzer imagery given in Figure \ref{fig:FindingChart} suggests that Nessie is at least two or three times longer than ``Nessie Classic," measuring at least $3^\circ$ long (``Nessie-Extended"). Very careful
inspection (pan and zoom Figure \ref{fig:FindingChart}) of the Spitzer images suggests
that Nessie \emph{could be} even longer.  If
one optimistically connects what appear to be all the relevant pieces
then ``Nessie Optimistic" could be as much as $8^\circ$ long (light white chalk line in Figure \ref{fig:FindingChart}). The optimism involved in seeing the longest extent for Nessie could  be warranted if bright star-forming regions have broken up the continuous extinction feature, and/or if the background emission fluctuates enough to make the extinction  hard to detect.

Determining the physical, three-dimensional, nature of extensions to the
Nessie cloud requires a detailed analysis of the velocity of the gas
associated with the dust responsible for mid-IR extinction. We
offer such an analysis below (\S\ref{3D}), but here we note that
if Nessie (as is nearly certain given its velocity range) lies in or
near the Scutum-Centaurus Arm of the Milky Way, then its distance is
roughly 3.1 kpc (cf. Jackson et al. 2010). At that distance, Nessie
Classic is roughly 80 pc long, Nessie Extended is 160 pc long, and
Nessie Optimistic is 430 pc long. For any of these lengths, the dark filament's
width is of order 0.01 degrees (0.5 pc), according to Jackson et al.'s (2010) analysis of the Spitzer imagery. Thus, clouds's axial ratio is about
150 for Nessie Classic, 300 for Nessie Extended, and nearly three times
more, 800, for Nessie Optimisitc. (These calculations are based on Table 1 a publicly-available interactive spreadsheet, at
\url{https://docs.google.com/spreadsheet/ccc?key=0AhIRxiTe1u6BdDlXOC10Zkd3WUNNZHVnRlhfeWhJYlE} a snapshot of which is shown as Figure \ref{fig:table1}.)