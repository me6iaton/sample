{\bf ABSTRACT} The very long, thin infrared dark cloud ``Nessie" is even longer than had been previously claimed, and an analysis of its Galactic location suggests that it lies directly in the Milky Way's mid-plane, tracing out a highly elongated bone-like feature within the prominent Scutum-Centaurus spiral arm. Re-analysis of mid-infrared imagery from the Spitzer Space Telescope shows that this IRDC is at least 2, and possibly as many as 8 times longer than had originally been claimed by Nessie's discoverers, \citet{Jackson2010}; its aspect ratio is therefore at least 150:1, and possibly as large as 800:1. A careful accounting for both the Sun's offset from the Galactic plane ($\sim 25$ pc) and the Galactic center's offset from the $(l^{II},b^{II})=(0,0)$ position defined by the IAU in 1959 shows that the latitude of the true Galactic mid-plane at the 3.1 kpc distance to the Scutum-Centaurus Arm is not $b=0$, but instead closer to $b=-0.5$, which is the latitude of Nessie to within a few pc. Apparently, Nessie lies {\it in} the Galactic mid-plane. An analysis of the radial velocities of low-density (CO) and high-density (${\rm NH}_3$) gas associated with the Nessie dust feature suggests that Nessie runs along the Scutum-Centaurus Arm in position-position-velocity space, which means it likely forms a dense `spine' of the arm in real space as well. No galaxy-scale simulation to date has the spatial resolution to predict a Nessie-like feature, but extant simulations do suggest that highly elongated over-dense filaments should be associated with a galaxy's spiral arms. Nessie is situated in the closest major spiral arm to the Sun toward the inner Galaxy, and appears almost perpendicular to our line of sight, making it the easiest feature of its kind to detect from our location (a shadow of an Arm's bone, illuminated by the Galaxy beyond). Although the Sun's ($\sim 25$ pc) offset from the Galactic plane is not large in comparison with the half-thickness of the plane as traced by Population I objects such as GMCs and HII regions ($\sim 200$ pc; \citet{2013A&ARv..21...61R}), it may be significant compared with an extremely thin layer that might be traced out by Nessie-like "bones" of the Milky Way. Future high-resolution extinction and molecular line data may therefore allow us to exploit the Sun's position above the plane to gain a (very foreshortened) view ``from above" of dense gas in Milky Way's disk and its structure. 
