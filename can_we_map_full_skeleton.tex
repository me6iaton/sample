\subsection{Can We Map the Full Skeleton of the Milky Way?}
\label{future}

In an ideal Universe, we would be able to travel far enough outside of the Milky Way to observe it from ``outside," the way that we see, for example, Andromeda.  The story for years has gone that generating an observationally-based plan (overhead) view of the Milky Way is impossible, because we are ``in" the Plane.  It is as if Earth-bound astronomers have been living in the 2D world Edwin A. Abbott famously called `Flatland \citep{Abbott2008} when it comes to thinking about how to image the Milky Way's spiral structure. But, we can escape Flatland by realizing that the tiny offset of the Sun above the Milky Way's midplane can give us a tiny, but useful, bit of perspective on the 3D structure of the Milky Way, and it can  offer a (highly-foreshortened!) overhead view.  This perspective is only useful when looking at very sharp, very narrow, features like Nessie, because puffier, more standard, arm-defining features will overlap too much to be separated in a very foreshortened view. 

Carry out the following thought experiment.  Draw a rough plan of a spiral galaxy on a very flat piece of paper.  Position a vantage point a tiny distance (a few hundredths of an inch) above that piece of paper, about two-thirds of the way out from the center of the galaxy.  Now give the observer at that vantage point super-sharp eyesight and ask if the observer can separate the  spiral arm features you drew, as they observe them.  They can--if and only if the spiral you drew has very narrow features defining its arms.  If the observer were exactly {\it in} the piece of paper (living in Flatland), separating the arms would be impossible, regardless of their width.  We are, like your observer, are at a tiny, tiny, elevation off of a spiral galaxy, and our vision is good enough to separate very skinny arm-like features.

So, how might we use out vantage point above the Plane to map out more of the Milky Way's skeleton?  It turns out that Nessie is located in a place where seeing a very long IRDC projected parallel to the Galactic Plane should be  easiest.  Look again at Figure \ref{fig:topview}, and consider Nessie's placement there.  According to the current (data-based cartoon) view of the Milky Way shown in Figure \ref{fig:topview}, Nessie is in the closest major spiral arm (Scutum-Centaurus) to us, along a direction toward, but not exactly toward, the (confusing) Galactic Center.  Nessie's placement there means that it will have a bright background illumination as seen from further out in the Galaxy (e.g. from the Sun), and that it will have a long extent on the Sky as compared with more distant or less perpendicular-to-our-line-of-sight objects.  It is always good when one finds what should be the easiest-to-see example of a new phenomenon first, so we are reassured that Nessie was the first ``Bone" of the Milky Way found.

To find more `Nessies,' if such narrow ``bone-like" features are in fact typical in spiral arms, we need to be clever about where and how we look. Our current understanding of the Milky Way's spatial and velocity structure will allow us to draw more velocity-encoded lines like the ones shown in Figure \ref{fig:coloredlines} on the Sky, mapping out the whole Galaxy as seen from the Sun's vantage point.  With such predictions in hand, we should design algorithms to look for dust clouds elongated (roughly) along those lines, and then we should examine the velocity structure of the elongated features, as we do in \S \ref{3D}, above.   Of course, we need to be flexible in which features we accept as possible other ``bones," remembering that the model we will use to draw the expected features on the Sky is the same one we seek to refine!  It is likely that a Bayesian approach, using the extant Milky Way model as a prior, will succeeed in this way.  

As extinction and dust emission maps cover more and more of the sky at ever-improving resolution and sensitivity, we should be able to map more and more of the Milky Way's skeleton.  New wide-field extinction-based efforts based at first on Pan-Starrs, and ultimately on Gaia, will be tremendously helpful in these efforts in the coming decade.

Recent (e.g. Spitzer, Herschel) mid- and far-infrared imagery already suggests that: 1) not all galaxies once thought to be dominated by a spiral pattern really are; and 2) not all IRDCs are likely to be part of the Milky Way's skeleton.  As mentioned above, images like Figure \ref{fig:IC342} clearly show that spiral galaxies can be very web-like, with long, straight filaments interconnecting spiral arms. Thus, some of the features seen as long, skinny, IRDCs in the Milky Way could very well {\it not} be part of spiral arms, even if they are part of a Galaxy-wide structural pattern.  This possibility will clearly complicate the modeling discussed above, but that just makes it more interesting! New data from ever-deeper and ever-sharper extragalactic observations will likely reveal even more complex galaxy strucutres.  Combining ALMA thermal dust emission and molecular line observations of slightly-inclined galaxies will allow us to combine structural image with velocity information facilitating ever-improving model comparison.

Over the past 15 years, since their discovery, there have been many efforts to catalog and characterize IRDCs, and it is clear that not all IRDCs are, or should be, part of Nessie-like bones.  

The catalog compiled by \citet{Peretto2009a} lists 11,000 IRDCs, but none of the features cataloged will be Nessie-like on its own.  The structure-finding algorithm used in the Peretto \& Fuller work is biased toward finding core-like roundish peaks, so a cloud like Nessie forms a connect-the-dots pattern in the Peretto \& Fuller catalog.  In fact, Nessie is comprised of $\sim 100$ Peretto \& Fuller sources. So, while the Peretto \& Fuller catalog is tremendously useful to the study of the properties of massive star forming cores, it will only become useful for finding ``bones" when someone applies a clever dot-connecting algorithm to it, and its ilk.

Some very large, and/or very extended, IRDCs, such as the so-called ``Massive Molecular Filament" studied by \citet{Battersby}, are not located along the plane-of-the-sky projected spiral arms.  These clouds, which are probably just massive star-forming regions near but not exactly in the Galactic plane, do not appear as straight or highly-elongated as Nessie (cf.the``wisp" discussed by \citet {2013A&A...559A..34L}), and they may offer a hint at what threshold to set in looking for elongated features as we search for more bones.  Futher, once numerical galaxy-simulation modeling resolution catches up to observational resolution, models should be able to say whether they other, non-bone-like, massive IRDCs had their origins long ago in bones, or form in some other way.