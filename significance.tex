\section{What is the Significance of Nessie-like structures within a Spiral Galaxy?}
\subsection{A Bone of the Galaxy}
\label{spine}
All the evidence presented in this paper, taken together, strongly suggests that Nessie forms a spine-like feature that runs down the center of the Scutum-Centaurus Arm of the Milky Way.  How did it get there?  Is it the crest of a classic spiral density wave \citep{Lin1964}, or does it have some other cause?  Any feature this long and skinny that is not controlled by Galactic-scale forces will be subject to a variety of instabilities, and cannot last long. It would be great if we could look to numerical simulations for answers, but today's simulations can, alas, only give hints.  Nessie is {\it so} skinny, and {\it so} much denser than its surroundings that no extant numerical simulation has the combination of spatial resolution and dynamic range in density needed to produce a feature like it.

Figure \ref{fig:simulation} offers a snapshot of a numerical simulation \cite{2013MNRAS.432..653D} that represents the state of the art at present (available as a movie at \url{http://empslocal.ex.ac.uk/people/staff/cld214/movies.html}).  One can see density features that are highly elongated, both within the spiral arms, and also between the arms.  Many of the features between the arms in Figure \ref{fig:simulation} are similar to the `spurs' and `feathers' that have been simulated and observed by E. Ostriker and colleagues \citep{Shetty2006,Vigne2008,Corder2008}.   Figure \ref{fig:IC342} (discussed below) shows a recent WISE image of the galaxy IC342 \citep{Jarrett2013}, and it is clear from that image that some `spiral' galaxies also exhibit inter-arm filaments that are even more pronounced than the simulated spurs and feathers.  

In the case of Nessie, the velocity information analyzed in \S \ref{CO} and \ref{ammonia} seems to very strongly favor Nessie's being oriented exactly along (within, as the backbone of) an arm (Scutum-Centaurus) over the idea that Nessie is a spur or interam filament.


Estimates for the mass of Nessie under various assumptions are given in Table 1.  Jackson et  al. 2010  model Nessie as a(n unmagnetized) self-gravitating fluid cylinder supported against collapse by a ``turbulent" analog of thermal pressure, undergoing the sausage instability discussed in \citet{1953ApJ...118..116C}. For the observed line width of HNC, the theoretical critical mass per unit length, $m_l$, is 525 ${\rm M}_\odot/{\rm pc}$ for Nessie.  But, if, as Jackson et al. explain, one estimates $m_l$ using HNC emission itself and (uncertain) abundance values for HNC, then $110 < m_l < 5 \times 10^4 {\rm M}_\odot {\rm pc}^{-1}$. Given that the low end of this range ($110 {\rm M}_\odot {\rm pc}^{-1}$, favored by Jackson et al.) gives a very low value for extinction toward Nessie ($A_V \sim 4$ mag), we favor higher values $m_l$, needed to be consistent with the observed IR extinction. Recent LABOCA observations of dust continuum emission from pieces of Nessie (Kauffmann, private communication), suggest that $m_l \simgreat 10^3$ in the mid-IR-opaque portions of Nessie.  
So, at present, it would appear that there is at least an order-of-magnitude uncertainty in $m_l$.  Some of this uncertainty is caused by the definition of Nessie's shape, which makes it unclear which ``mass" to measure in calculating $m_l$, but more is due to the vagaries of converting molecular line emission and/or dust continuum to true masses. 

It is not the goal of this paper to produce a more definitive estimate of Nessie's $m_l$ or total mass, or to model Nessie's internal density structure.  Instead, here, we only seek to estimate the total mass of Nessie in order to consider its mass as a fraction of that in the Galaxy or in a spiral arm.  So, Table 1 offers rough estimates of the mass of cylinders, whose (constant) average density is set so that the typical extinctions associated with Nessie's IR-dark ($A_v\sim 100$) and HCN bright ($A_V \simgreat$ a few mag) radii are sensible.  Assuming a mean density for the mid-IR opaque material of $10^5$ cm$^{-3}$, then Nessie Classic is $1 \times 10^5$ M$_\odot$, Nessie Extended is $2 \times 10^5$ M$_\odot$ and Nessie Optimistic is $5 \times 10^5$ M$_\odot$.  If one assumes that the envelope traced by the HNC observations of Jackson et al. (2010) for Nessie Classic continues along Nessie's length, then the mass of a $n\sim 500$ cm$^{-3}$ cylindrical tube (see Table 1) associated with Nessie would be $5 \times 10^4$ M$_\odot$ for Classic and $3 \times 10^5$ M$_\odot$ for Optimistic.  For the Optimistic case, this mass amounts to 2 millionths of the total baryonic mass (assuming $\sim 10^{11}$ M$_\odot$ total) of the Milky Way.  To use this fraction in order to estimate the total number of ``Nessies" discoverable in the Milky Way's ISM, we need to remember that the HCN mass is likely a lower limit (meaning the mass fraction is an upper limit), and that most of the gas mass in the ISM is at low density, following a log-normal-like density distribution.  Given those caveats, we estimate that of order thousands of additional Nessie-like features should be discoverable, if they are characteristic of spiral arms.
