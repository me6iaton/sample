\subsubsection{NH$_3$ Velocities}
\label{ammonia}
To estimate the 3D orientation of Nessie more precisely, we need to employ a gas tracer whose emission is sparser than CO's in position-position-velocity space. Many recent studies have shown that IRDCs typically host over-dense blobs of gas (often called ``clumps" or ``cores") that provide the gaseous reservoirs for the formation for massive stars.  Thus, several studies have been undertaken to survey IRDCs and their ilk for emission in molecular lines that trace high-density ($\gg 10^3$ cm$^{-3}$), potentially star-forming, gas.  

The H$_2$O Southern Galactic Plane (HOPS) Survey \citep{Purcell2012b} has surveyed hundreds of sites of massive star formation visible from the Southern Hemisphere for ${\rm NH}_3$ emission, which traces gas at densities $n\gtrsim 10^4$ cm$^{-3}$.   The HOPS targets were selected based on H$_2$O maser emission, thermal molecular emission, and radio recombination lines, so as to include nearly all known regions of massive star formation within the surveyed area.  These ``massive-star-forming region" selection criteria mean that the HOPS database includes ${\rm NH}_3$ spectra for dozens of positions within the longitude range covered by Nessie.   

Figure \ref{fig:HOPSoverlay} shows an overlay of HOPS sources' ${\rm NH}_3$-determined LSR velocities on the Spitzer image of Nessie used in Figure \ref{fig:coloredlines}.  The (color-coded) velocities of the HOPS sources, for both Nessie Classic, and Nessie Extended (see Figure \ref{fig:FindingChart}), agree remarkably well with what is predicted for the Scutum-Centaurus Arm  (color-coded lines). Note that agreement of the  ${\rm NH}_3$ and predicted velocity to within 5 km s$^{-1}$ is indicated by light-colored circles around the HOPS symbol (see caption for details).  White circles correspond to the Nessie Extended sources also shown in Figure \ref{fig:pvdiagram}, below, and grey circles mark other points, in Nessie Optimistic, also likely (based on their velocity) to be associated with the Scutum-Centaurus arm.  The velocities of sources at latitudes much different from Nessie's within this longitude range largely do {\it not} agree, and those sources are unlikely to associated with the near-side of the Scutum-Centaurus Arm.    

For Nessie Classic, Jackson et al. (2010) had already noted a very narrow velocity range for dense gas associated with the IRDC, based on HNC observations.   What is new here is the three-dimensional (latitude, longitude, \textit{and} velocity) association of a {\it longer} Nessie's dense gas with predictions for where the centroid of the Milky Way's Scutum-Centaurus Arm's ``middle" would lie.  

Figure \ref{fig:pvdiagram}, which offers a position-velocity diagram of CO (color) and ${\rm NH}_3$ emission (black dots) together, shows the association of the Nessie-HOPS sources with the Scutum Centaurus Arm most clearly.  What is most remarkable about Figure \ref{fig:pvdiagram} is that the black line sloping through the figure is {\it not} a fit to the black dots representing the HOPS sources.  Instead, that line indicates the position-velocity trace of the Scutum-Centaurus Arm based on \citep{Dame2011} data for the full Galaxy, not just this small longitude range.   Figure \ref{fig:pvdiagram} implies that Nessie goes right down the ``spine" of the Scutum-Centaurus Arm, as best we can measure its position in CO position-velocity space.

